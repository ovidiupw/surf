În terminologia grafurilor, frontiera reprezintă lista nodurilor nevizitate \footnote{"[...] unexpanded (unvisited) nodes." \cite{PantSrinivasanMenczer} }. Pentru un crawler web, frontiera reprezintă o listă de adrese web neparcurse. Procesul de căutare pornește de la o adresă numită \textit{rădăcină}, care este preluată din web și procesată, urmând ca link-urile găsite să fie adăugate în frontieră. Această listă poate căpăta dimensiuni foarte mari încă de la începutul procesului de parcurgere a paginilor, întrucât media numărului de legături de pe o pagină web este de 7 link-uri/pagină \footnote{"The web  may  be  viewed  as  a  directed  graph  in  which each vertex is a static HTML web page, and each edge is a hyperlink from one web page to another. Current estimates suggest that this graph has roughly a billion vertices, and an average degree of about 7."\cite{StochasticModels}}. Procesul se repetă, alegându-se, folosind un anumit algoritm, link-uri din frontieră, până când frontiera se golește, sau altă condiție de oprire permite finalizarea căutării.
\\