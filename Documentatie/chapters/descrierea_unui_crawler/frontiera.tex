In terminologia grafurilor, frontiera reprezinta lista nodurilor nevizitate \footnote{"[...] unexpanded (unvisited) nodes." \cite{PantSrinivasanMenczer} }. Pentru un crawler web, frontiera reprezinta o lista de likuri neparcurse. Procesul de cautare porneste de la un link numit radacina, care este preluat din web si procesat, uramand ca link-urile gasite sa fie adaugate in frontiera. Aceasta lista poate capata dimensiuni foarte mari inca de la inceputtul procesului de parcurgere a paginilor, intrucat media de legaturi de pe o pagina web este 7 link-uri/pagina \footnote{"The web  may  be  viewed  as  a  directed  graph  in  which each vertex is a static HTML web page, and each edge is a hyperlink from one web page to another. Current estimates suggest that this graph has roughly a billion vertices, and an average degree of about 7."\cite{StochasticModels}}. Procesul se repeta, alegandu-se, folosind un algoritm, link-uri din frontiera, pana cand frontiera se goleste sau alta conditie de oprire permite finalizarea cautarii.
\\