Sistemul de pluginuri (prezentat, la nivel abstract, în \textit{"Figura 9"}), reprezintă o modalitate de a adăuga crawler-ului web "Surf" funcționalități legate de procesarea informațiilor preluate din siturile web vizitate. Plugin-urile vor fi dezvoltate ca funcții Lambda, care se vor afla în contul AWS al proprietarului crawler-ului. Fiecare plugin va adera la o interfață, cu scopul realizării schimbului de informații în cadrul automatului finit definit prin "AWS Step Functions". Totodată, fiecare plugin va specifica dacă în cadrul execuției sale vor fi salvate date și metadate în sistemele persistente de stocare. Sistemul va putea genera rezultate ale parcurgerii informațiilor din paginile web vizitate la oricare pas al execuției paralele din cadrul automatului finit, datele fiind etichetate cu prefixul funcției  care le-a generat, pentru a facilita selectarea și agregarea acestora. Cu alte cuvinte, fiecare task paralel din cadrul automatului finit va trece printr-o serie de stagii configurabile în ceea ce priveste extragerea și prelucrarea informațiilor provenite de la stagiul precedent. În cazul în care nu se dorește specificarea de plugin-uri, atunci execuția taskurilor de crawling din automatul finit va avea comportamentul de bază descris în capitolul \textit{"Crawling în cloud"}.