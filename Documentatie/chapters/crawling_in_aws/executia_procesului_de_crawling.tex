Această secțiune prezintă, în detaliu, modul în care aplicația "Surf" îndeplinește procesul de parcurgere a siturilor web. Mai exact, se descriu mecanismele prin care, având la dispoziție infrastructura creată cu ajutorul generatorului de resurse, utilizatorul poate beneficia de rezultatele extrase din paginile web parcurse de catre crawler. Pentru simplitate, ne vom referi la ansamblul acestor mecanisme drept un \textit{workflow}\footnote{Din engl. "workflow", care se traduce, în acest context, drept un flux de lucru, sau un proces în cadrul căruia se desfășoară o serie bine definită de activități (i.e. execuții de programe)}.
\\

\noindent
Un \textit{workflow} reprezintă unitatea de bază pe care crawler-ul web "Surf" o poate interpreta. Dintr-un \textit{workflow} se pot genera oricâte execuții. Începerea execuției unui \textit{workflow} trebuie să fie precedată de crearea acestuia. O execuție a unui \textit{workflow} întreprinde următoarele acțiuni, cu scopul parcurgerii paginilor web pentru extragerea informațiilor:

\begin{enumerate}

	\item{Preluarea, validarea și interpretarea datelor de configurare a procesului de crawling;}
	
	\item{Stocarea persistentă a definiției unei execuții a unui \textit{workflow} în baza de date, cu scopul accesării datelor necesare în contextul rulării crawler-ului;}
	
	\item{Inițializarea unui automat finit și determinist, în vederea modelării parcurgerii recursive a siturilor web în manieră \textit{breadth-first} \footnote{Parcurgea în manieră \textit{breadth-first} asigură vizitarea fiecărui nod dintr-un arbore de la adâncimea curentă, înainte de a trece la parcurgerea nodurilor de la următorul nivel de adâncime}, prin "AWS Step Functions";}
	
	\item{Crearea sarcinilor pentru parcurgerea paginilor web (\textit{taskuri}\footnote{În contextul execuției unui \textit{workflow}, un \textit{task} reprezintă o unitate de lucru, sub forma unei înregistrări în baza de date, ce poate fi preluată și executată de către un program aflat în execuție (e.g. o funcție Lambda)}), înregistrarea datelor și metadatelor rezultate în urma parcurgerii, precum și a datelor ce modelează execuția procesului de crawling (e.g. timpi maximi de răspus, adrese URL vizitate deja etc.);}
	
	\item{Distribuirea \textit{taskurilor} catre funcțiile Lambda (\textit{workers}\footnote{Un \textit{worker} reprezintă un program executabil. În aplicatia "Surf", funcțiile Lambda reprezintă \textit{workerii}}) ce le vor prelua și executa;}
	
	\item{Executarea funcțiilor Lambda și stocarea/intoarcerea rezultatelor;}
	
	\item{Sincronizarea execuțiilor paralele ale funcțiilor Lambda și centralizarea rezultatelor în urma execuției \textit{task-urilor};}
	
	\item{Pornirea recursivă a unui nou \textit{workflow}, cu scopul de a parcurge un nou nivel de adâncime în procesul de crawling și de a gestiona gradul de încărcare a sistemului distribuit (e.g. respectarea limitărilor impuse de serviciile AWS, precum numărul maxim de execuții concurente ale funcțiilor Lambda în cadrul execuției unui automat "AWS Step Functions").}
	
\end{enumerate} 

