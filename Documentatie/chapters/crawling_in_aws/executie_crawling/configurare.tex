Pentru inceperea executiei unui workflow, este necesara, in prealabil, crearea acestuia. \textit{"Tabelul 1"} descrie datele necesare crearii unui workflow, date dintre care o parte sunt configurabile de catre utilizator (i.e. campul "Config."\footnote{prescurtare de la substantivul \textit{Configurabil}} are valoarea \textit{DA}), iar o parte sunt memorate in mod implicit odata ce workflow-ul este creat. Suita de configurari descrisa in \textit{"Tabelul 1"} coexista cu seria de configurari generate de creatorul de resurse. Diferenta intre cele doua consta in faptul ca generatorul de resurse defineste configurari la nivel de cont AWS. Impartirea configurarilor in doua nivele de abstractizare (i.e. scazut pentru cele de la nivelul resurselor din contul AWS si ridicat pentru cele oferite in etapa de creare a unui workflow) confera flexibilitate in utilizarea crawler-ului "Surf". Pentru a evidentia contrastul intre cele doua configurari, vom enumera, in continuare, cateva elemente esentiale ale configurarii la nivel de resurse din contul AWS:

\begin{itemize}
	\item{timpul maxim de asteptare pentru ca o cerere cloud sa fie satisfacuta;}
	\item{regiunea in care sunt create resursele AWS;}
	\item{numarul maxim de cereri pe secunda admis de catre API;}
	\item{numarul maxim de cereri pe secunda admise pentru tabelele din baza de date (configurare importanta pentru controlul costului);}
	\item{stabilirea nivelului de logare (i.e. urmarire a actiunilor utilizatorilor in cadrul interactiunii cu API-ul "Surf);}
	\item{stabilirea limbajului in care va fi generat clientul pentru API-ul creat prin API Gateway.}

\end{itemize}

\begin{table}[h]
	\centering
    \begin{tabular}{|M{2.5cm}|M{1.6cm}|M{1.25cm}|M{7.25cm}|}
    	\hline 
    	Nume camp & Tip & Config. & Descriere \\ \hline
    	
    Id & Text & NU & Id atribuit workflow-ului pentru a putea fi identificat in mod unic printre celelalte workflow-uri din tabela (din baza de date) care le gazduieste \\ \hline
    
    Data Creare & Numeric & NU & Data crearii, reprezentand numarul de milisecunde de la 1970 (engl. "epoch milliseconds") \\ \hline
    
    Proprietar & Text & NU & Identitatea atribuita utilizatorului ce a creat workflow-ul. Necesar pentru a stabili permisiuni de acces. \\ \hline
    
    Nume & Text & DA & Nume familiar atribuit workflow-ului pentru identificare facila \\ \hline
    
    Adresa Inceput & Text & DA & URL-ul de la care va porni procesul de crawling \\ \hline
    
    Selector URL & Text & DA & Expresie regulata ce valideaza daca un URL de pe o pagina vizitata de catre crawler este adaugat recursiv la lista sarcinilor pentru crawling \\ \hline
    
    Adancime maxima & Numeric & DA & Adancimea maxima la care vor fi parcurse paginile web de catre crawler (i.e. in arborele de parcurgere recursiva) \\ \hline
    
    Pagini pe nivel & Numeric & DA & Numarul maxim de pagini ce vor fi parcurse de catre crawler pe un anumit nivel de adancime (i.e. in arborele de parcurgere recursiva) \\ \hline
    
    Dimensiune maxima pagina & Numeric & DA & Dimensiunea maxima admisa (in octeti) pentru ca un sit web sa fie parcurs \\ \hline
    
    Grad de paralelizare & Numeric & DA & Numarul maxim de crawleri concurenti dintr-un workflow \\ \hline
    
    Politica de selectie & JSON & DA & Definitie folosita pentru a extrage date din paginile web parcurse, in functie de anumiti parametri \\ \hline
    
    Politica de reincercare & JSON & DA & Definitie folosita pentru a stabili cazurile in care incercarile esuate de a parcurge paginile web trebuie reincercate \\ \hline
    
    \end{tabular}
    \caption{Definitia unui workflow}
\end{table}
\clearpage