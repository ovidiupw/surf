Pentru începerea execuției unui workflow, este necesară, în prealabil, crearea acestuia. \textit{"Tabelul 1"} descrie datele necesare creării unui workflow, date dintre care o parte sunt configurabile de către utilizator (i.e. câmpul "Config."\footnote{prescurtare de la substantivul \textit{Configurabil}} are valoarea \textit{DA}), iar o parte sunt memorate în mod implicit odată ce workflow-ul este creat. Suita de configurări descrisă în \textit{"Tabelul 1"} coexistă cu seria de configurări generate de creatorul de resurse. Diferența între cele două constă în faptul că generatorul de resurse definește configurări la nivel de cont AWS. Împărțirea configurărilor în două nivele de abstractizare (i.e. scăzut pentru cele de la nivelul resurselor din contul AWS și ridicat pentru cele oferite în etapa de creare a unui workflow) conferă flexibilitate în utilizarea crawler-ului "Surf". Pentru a evidenția contrastul între cele două configurări, vom enumera, în continuare, câteva elemente esențiale ale configurării la nivel de resurse din contul AWS:

\begin{itemize}
	\item{timpul maxim de așteptare pentru ca o cerere cloud să fie satisfacută;}
	\item{regiunea în care sunt create resursele AWS;}
	\item{numărul maxim de cereri pe secundă admis de catre API;}
	\item{numărul maxim de cereri pe secundă admise pentru tabelele din baza de date (configurare importantă pentru controlul costului);}
	\item{stabilirea nivelului de jurnalizare (i.e. urmărire a acțiunilor utilizatorilor în cadrul interacțiunii cu API-ul "Surf);}
	\item{stabilirea limbajului în care va fi generat clientul pentru API-ul creat prin API Gateway.}

\end{itemize}

\begin{table}[h]
	\centering
    \begin{tabular}{|M{2.5cm}|M{1.6cm}|M{1.25cm}|M{7.25cm}|}
    	\hline 
    	Nume câmp & Tip & Config. & Descriere \\ \hline
    	
    Id & Text & NU & Id atribuit workflow-ului pentru a putea fi identificat în mod unic printre celelalte workflow-uri din tabela (din baza de date) care le găzduiește \\ \hline
    
    Dată Creare & Numeric & NU & Data creării, reprezentând numărul de milisecunde de la 1970 (engl. "epoch milliseconds") \\ \hline
    
    Proprietar & Text & NU & Identitatea atribuită utilizatorului ce a creat workflow-ul. Necesar pentru a stabili permisiuni de acces. \\ \hline
    
    Nume & Text & DA & Nume familiar atribuit workflow-ului pentru identificare facilă \\ \hline
    
    Adresă Început & Text & DA & URL-ul de la care va porni procesul de crawling \\ \hline
    
    Selector URL & Text & DA & Expresie regulată ce validează dacă un URL de pe o pagină vizitată de către crawler este adăugat recursiv la lista sarcinilor pentru crawling \\ \hline
    
    Adâncime maximă & Numeric & DA & Adâncimea maximă la care vor fi parcurse paginile web de către crawler (i.e. în arborele de parcurgere recursivă) \\ \hline
    
    Pagini pe nivel & Numeric & DA & Numărul maxim de pagini ce vor fi parcurse de către crawler pe un anumit nivel de adâncime (i.e. în arborele de parcurgere recursivă) \\ \hline
    
    Dimensiune maximă pagină & Numeric & DA & Dimensiunea maximă admisă (în octeți) pentru ca un sit web să fie parcurs \\ \hline
    
    Grad de paralelizare & Numeric & DA & Numărul maxim de crawleri concurenți dintr-un workflow \\ \hline
    
    Politica de selecție & JSON & DA & Definiție folosită pentru a extrage date din paginile web parcurse, în funcție de anumiți parametri \\ \hline
    
    Politica de reîncercare & JSON & DA & Definiție folosită pentru a stabili cazurile în care încercările eșuate de a parcurge paginile web trebuie reîncercate \\ \hline
    
    \end{tabular}
    \caption{Definiția unui workflow}
\end{table}
\clearpage