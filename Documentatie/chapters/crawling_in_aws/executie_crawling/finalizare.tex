Procesul de finalizare a unei sesiuni de crawling are drept obiective centralizarea și analizarea datelor provenite în urma executării, în paralel, a funcțiilor responsabile pentru parcurgerea paginilor web. Responsabilitățile funcției de finalizare se pot împărți în două categorii, în funcție de motivul ce a determinat invocarea acesteia:

\begin{enumerate}
	\item{Finalizarea unei execuții îndeplinite cu succes, caz în care nu au intervenit erori extraordinare (i.e. netratate) în cadrul cel puțin unuia dintre task-urile executate în paralel;}
	\item{Finalizarea unei sesiuni de parcurgere în care au existat erori în cadrul execuției cel puțin unuia dintre task-urile executate în paralel, caz în care funcția de finalizare va îndeplini toate sarcinile ce au ramas neîndeplinite în urma terminării bruște a execuției sarcinilor paralele de crawling.}
\end{enumerate} 

\noindent
Indiferent de situația în care se află execuția worfklow-ului, funcția de finalizare a sesiunii de crawling are în vedere înregistrarea metadatelor cu privire la progresul workflow-ului (e.g. metrici Cloudwatch) și luarea deciziei conform căreia invocarea recursivă a următoarei sesiuni de crawling se face începand cu același nivel de adâncime sau cu unul superior (i.e. mai mare), în concordanță cu numarul maxim de pagini ce poate fi vizitat pe un anumit nivel de adâncime, de catre crawler.

