Aplicația "Surf" este construită și rulează în cloud. Dezvoltarea unei aplicații pe suportul unei infrastructuri cloud este substanțial diferită față de abordarea clasică, pe o singură mașină de calcul. Cloud computing-ul pune la dispoziție, la cerere, un set de resurse (e.g. putere de procesare, spațiu de stocare) pe care un utilizator (e.g. aplicația "Surf") le poate folosi. Faptul că toate aceste resurse sunt administrate de fiecare serviciu cloud în parte simplifică sarcinile consumatorului: se pot solicita mai multe sau mai puține resurse, după gradul de încărcare al aplicației dezvoltate în cloud. Acest lucru se realizează într-o manieră transparentă pentru utilizator, fără a necesita efort suplimentar din partea acestuia.
\\
\\
Dezvoltarea aplicației "Surf" folosind servicii cloud mai prezintă un avantaj considerabil, deoarece comunicarea prin Internet\footnote{Protocolul de comunicare folosit este \textit{Transmission Control Protocol}\cite{tcp}, întrucat acesta garantează transmiterea nealterată a datelor între două noduri ale rețelei} între componentele aplicației și, implicit, utilizarea unui format agnostic de arhitectură sau limbaj de programare, conduce la un grad mare de decuplare. Fiind decuplate, componentele aplicației pot fi construite, testate și depanate\footnote{\emph{engl.} debugged} în izolare și pot funcționa ca module (plugin-uri) în alcătuirea arhitecturii sistemului.
\\
\\
A dezvolta un serviciu web folosind exclusiv tehnologii cloud permite realizarea unui sistem de subscripții granular. Un utilizator al aplicației "Surf" poate solicita una sau mai multe instanțe ale crawler-ului care să ruleze în paralel, în funcție de necesitățile proprii. Serviciile cloud permit aplicației să scaleze orizontal ca timp, permițând utilizatorului realizarea mai multor sarcini de parcurugere automată web (deci, în consecință, acoperirea unui volum mai mare de informații) în același interval de timp.
\\
\\
Un alt aspect relevant în realizarea serviciului web "Surf" este caracterul său \emph{serverless}. Infrastructura \emph{serverless} implică execuția codului sursă al programului în cadrul serviciilor cloud folosite pentru dezvoltarea aplicației. Cererile de execuție a codului sunt monetizate conform unei măsuri abstracte corespunzătoare resurselor utilizate pentru satisfacerea solicitării, ceea ce diferă față de modelul clasic, în care ar fi trebuit achiziționat hardware în acest scop\cite{serverless-characterization}.