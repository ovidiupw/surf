Aplicatia "Surf" este construita si ruleaza in cloud. Dezvoltarea unei aplicatii pe suportul unei infrastructuri cloud este substantial diferita fata de abordarea clasica, pe o singura masina de calcul. Cloud computing-ul pune la dispozitie, la cerere, un set de resurse (e.g. putere de procesare, spatiu de stocare) pe care un utilizator (e.g. aplicatia "Surf") le poate folosi. Faptul ca toate aceste resurse sunt administrate de fiecare serviciu cloud in parte usureaza sarcinile consumatorului: se pot solicta mai multe sau mai putine dupa gradul de incarcare al aplicatiei dezvoltate in cloud. Acest lucru se realizeaza intr-o maniera transparenta pentru utilizator, fara a necesita efort suplimentar din partea acestuia.
\\
\\
Dezvoltarea aplicatiei "Surf" folosind servicii cloud mai prezinta un avantaj considerabil, deoarece comunicarea prin internet intre componentele aplicatiei si, implicit, utilizarea unui format agnostic de arhitectura sau limbaj de programare, conduce la un grad mare de decuplare. Fiind decuplate, componentele aplicatiei pot fi construite, testate si depanate\footnote{\emph{engl.} debugged} in izolare si pot functiona ca module (plugin-uri) in alcatuirea arhitecturii sistemului.
\\
\\
A dezvolta un serviciu web folosind exclusiv tehnologii cloud permite realizarea unui sistem de subscriptii granular. Un utilizator al aplicatiei "Surf" poate solicita una sau mai multe instante ale crawler-ului care sa ruleze in paralel, in functie de necesitatile proprii. Serviciile cloud permit aplicatiei sa scaleze\footnote{https://en.wikipedia.org/wiki/Scalability} orizontal ca timp, permitand utilizatorului realizarea mai multor sarcini de parcurugere automata web (deci, in consecinta, acoperirea unui volum mai mare de informatii) in acelasi interval de timp.
\\
\\
Un alt aspect relevant in realizarea serviciului web "Surf" este caracterul sau \emph{serverless}. Infrastructura \emph{serverless} implica executia codului sursa al programului in cadrul serviciilor cloud folosite pentru dezvoltarea aplicatiei. Cererile de executie a codului sunt monetizate conform unei masuri abstracte corespunzatoare resurselor utilizate pentru satisfacerea solicitarii, ceea ce difera fata de modelul clasic, in care ar fi trebuit achizitionat hardware in acest scop\cite{serverless-characterization}.