La nivelul internetului, traficul global a crescut de aproximativ 850 de ori in perioada 2000 - 2015 \cite{cisco-internet-traffic}. World wide web-ul reprezinta o parte semnificativa a volumului de informatii interschimbate pe internet. In anul 2015 existau peste jumatate de miliard de situri web accesibile \cite{http://www.internetlivestats.com/total-number-of-websites/}. Fiecare pagina web raspunde anumitor nevoi (sociale, financiare, educationale etc.). O singura sursa de informatii este uneori suficienta pentru a raspunde nevoilor unui utilizator. Alteori, este necesar un ansamblu de surse informative (e.g. newsletter-e zilnice din 5 surse diferite), pentru a urmari un subiect din mai multe perspective sau a-i intregi continutul.
\\
\\
Prezenta lucrare urmareste elaborarea unui serviciu web distribuit in cloud pentru parcurgerea, selectarea, colectarea si indexarea informatiilor la nivelul world wide web-ului. Serviciul se adreseaza acelor persoane care vor sa automatizeze sarcinile de extragere a informatiilor web si garanteaza usurinta in utlizare, flexibilitate, extensibilitate si control precis asupra costurilor.
