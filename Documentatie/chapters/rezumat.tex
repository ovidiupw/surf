La nivelul Internetului, traficul global a crescut de aproximativ 850 de ori în perioada 2000 - 2015 \cite{cisco-internet-traffic}. World wide web-ul reprezintă o parte semnificativă a volumului de informaţii interschimbate pe Internet. În anul 2015 existau peste jumătate de miliard de situri web accesibile \cite{http://www.internetlivestats.com/total-number-of-websites/}. Fiecare pagină web răspunde anumitor nevoi (sociale, financiare, educaţionale etc.). O singură sursă de informaţii este, uneori, suficientă pentru a răspunde nevoilor unui utilizator. Alteori, este necesar un ansamblu de surse informative (e.g. newsletter-e zilnice din 5 surse diferite), pentru a urmări un subiect din mai multe perspective sau a-i întregi conţinutul.
\\
\\
Prezenta lucrare urmăreşte elaborarea unui serviciu web distribuit în cloud pentru parcurgerea, selectarea, colectarea şi indexarea informaţiilor la nivelul world wide web-ului. Serviciul se adresează acelor persoane care vor să automatizeze sarcinile de extragere a informaţiilor web şi garantează uşurinţă în utlizare, flexibilitate, extensibilitate şi control precis asupra costurilor.
