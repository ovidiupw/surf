Sistemul de pluginuri (prezentat, la nivel abstract, in \textit{"Figura 9"}), reprezinta o modalitate de a adauga crawler-ului web "Surf" functionalitati legate de procesarea informatiilor preluate din siturile web vizitate. Plugin-urile vor fi dezvoltate ca functii Lambda care se vor afla in contul AWS al proprietarului crawler-ului. Fiecare plugin va adera la o interfata cu scopul realizarii schimbului de informatii in cadrul automatului finit definit in cadrul "AWS Step Functions". Totodata, fiecare plugin va specifica daca, in cadrul executiei sale, vor fi salvate date si metadate in sistemele persistente de stocare. Sistemul va putea genera rezultate ale parcurgerii informatiilor din paginile web vizitate la oricare pas al executiei paralele din cadrul automatului finit, datele fiind etichetate cu prefixul functiei  care le-a generat, pentru a facilita selectarea si agregarea acestora. Cu alte cuvinte, fiecare task paralel din cadrul automatului finit va trece printr-o serie de stagii configurabile in ceea ce priveste extragerea si prelucrarea informatiilor provenite de la stagiul precedent. In cazul in care nu se doreste specificarea de plugin-uri, atunci executia taskurilor de crawling din automatul finit va avea comportamentul de baza, descris in capitolul \textit{"Crawling in cloud"}.