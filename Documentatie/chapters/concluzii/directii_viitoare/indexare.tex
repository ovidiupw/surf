Odata ce crawler-ul web "Surf" extrage datele din paginile web vizitate, acesta creaza si o serie de metadate, cu scopul indexarii locatiei datelor in S3. Desi acest sistem functioneaza, el este limitat la accesarea datelor despre localizarea informatiilor salvate. O imbunatatire a functionalitatii aplicatiei "Surf" ar reprezenta crearea unui sistem de indexare periodica a informatiilor. Un astfel de sistem ar asigura mecanisme complexe de localizare si extragere a datelor, atat in functie de sintaxa, cat si de semantica, indeplinind urmatoarele functii:

\begin{itemize}
	\item{indexare periodica, prin mecanismul de generare de evenimente oferit de "AWS CloudWatch Events", a datelor generate in procesul de parcurgere a siturilor web;}
	
	\item{Utilizarea unui framework pentru procesarea limbajului natural\footnote{engl. "Natural language processing" pentru extragerea metadatelor referitoare la datele textuale extrase in procesul de crawling;}}
	
	\item{Utilizarea functiilor de procesare a web-ului semantic\footnote{https://www.w3.org/standards/semanticweb/} pentru a extrage, pe baza unui algoritm de clusterizare\footnote{http://www.dictionary.com/browse/clustering}, structuri semantice definitorii din paginile web vizitate\footnote{http://schema.org/}.}	
	
\end{itemize}

Odata grupate si generate, datele ce alcatuiesc mecanismul de indexare pot fi clasificate utilizand algoritmi de clasificare probabilista, pentru a restrage spatiul de cautare atunci cand se recurge la cuvinte cheie.