\newcommand{\mimeDescription}{https://www.iana.org/assignments/media-types/media-types.xhtml}
\newcommand{\urlNormalization}{https://tools.ietf.org/html/rfc3986\#section-6}

Selecția informațiilor necesită parcurgerea siturilor web aflate la adresele URL pe care crawler-ul le are în vedere, procesarea, analizarea și prelucrarea informațiilor în funcție de tipul lor (e.g. html, xml, json, text) și extragerea blocurilor de conținut aferente.
\\
\\
Un "bloc de conținut" asociat unui URL reprezintă o parte a întregului conținut aflat la URL-ul respectiv, filtrată după anumite caracteristici stabilite de utilizatorul serviciului web. Aceste caracteristici pot fi:

\begin{itemize}
	\item{Pentru fișiere HTML/XML:
		\begin{itemize}
			\item{selectori CSS/jQuery \cite{css-selectors}}
			%\item{cuvinte cheie}
			%\item{o combinatie logica a punctelor de mai sus (e.g. toate tagurile \textless{}p\textgreater{ }  in care se afla cuvintele cheie "om" si "luna")}
		\end{itemize}			
	}
	\item{Pentru fișiere text:
		\begin{itemize}
			\item{cuvinte cheie}
		\end{itemize}			
	}
\end{itemize}

\noindent
Pentru tipurile de conținut enumerate mai sus sau pentru alte tipuri de conținut aflate la adresele URL vizitate de crawler, utilizatorul poate defini filtre bazate pe conținutul textual al URL-ului. Spre exemplu, se pot evita toate URL-urile care nu satisfac o anumita expresie regulată. %sau un anumit tip de continut  MIME\footnote{\mimeDescription} (e.g. "image/png"). 
\\
\\
Evitarea selectării informațiilor duplicate se ameliorează prin normalizarea\footnote{\urlNormalization} URL-urilor parcurse de către crawler.
