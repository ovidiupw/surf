\newcommand{\mimeDescription}{https://www.iana.org/assignments/media-types/media-types.xhtml}
\newcommand{\urlNormalization}{https://tools.ietf.org/html/rfc3986\#section-6}

Selectia informatiilor necesita parcurgerea siturilor web aflate la adresele URL pe care crawler-ul le are in vedere, parsarea informatiilor in functie de tipul lor (e.g. html, xml, json, text) si extragerea blocurilor de continut aferente.
\\
\\
Un "bloc de continut" asociat unui URL reprezinta o parte a intregului continut aflat la URL-ul respectiv, filtrata dupa anumite caracteristici stabilite de utilizatorul serviciului web. Aceste caracteristici pot fi:

\begin{itemize}
	\item{Pentru fisiere HTML/XML:
		\begin{itemize}
			\item{selectori CSS/jQuery}
			%\item{cuvinte cheie}
			%\item{o combinatie logica a punctelor de mai sus (e.g. toate tagurile \textless{}p\textgreater{ }  in care se afla cuvintele cheie "om" si "luna")}
		\end{itemize}			
	}
	\item{Pentru fisiere text:
		\begin{itemize}
			\item{cuvinte cheie}
		\end{itemize}			
	}
\end{itemize}

\noindent
Pentru tipurile de continut enumerate mai sus sau pentru alte tipuri de continut aflate la adresele URL vizitate de crawler, utilizatorul poate defini filtre bazate pe continutul textual al URL-ului. Spre exemplu, se pot evita toate URL-urile care nu satisfac o anumita expresie regulata. %sau un anumit tip de continut  MIME\footnote{\mimeDescription} (e.g. "image/png"). 
\\
\\
Evitarea selectarii informatiilor duplicate se amelioreaza prin normalizarea\footnote{\urlNormalization} URL-urilor parcurse de catre crawler.
