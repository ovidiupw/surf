\newcommand{\xmlDescription}{Extensible Markup Language - https://www.w3.org/XML/}
\newcommand{\jsonDescription}{JavaScript Object Notation - http://www.json.org/}
\newcommand{\crawlDescription}{Web crawler - https://www.techopedia.com/definition/10008/web-crawler}
\newcommand{\restDescription}{REST - http://www.ics.uci.edu/~fielding/pubs/dissertation/rest\_arch\_style.htm}
\newcommand{\awsDescription}{Amazon Web Services - https://aws.amazon.com}

Un serviciu web reprezintă o componentă funcțională ce îndeplinește anumite sarcini. Comunicarea cu un serviciu web se realizează independent de platformă, limbajul de programare sau sistemul de operare pe care este dezvoltat. Schimbul de informații se realizează prin mesaje text ce respectă un format standardizat, precum XML\footnote{\xmlDescription} sau JSON\footnote{\jsonDescription}. 
\\
\\
Aplicatia "Surf" reprezintă un serviciu web specializat în web crawling\footnote{\crawlDescription}, dezvoltat folosind tehnologii cloud din cadrul Amazon Web Services. Se urmărește crearea unui serviciu web cu disponibilitate permanentă, scalabil și de înaltă putere computațională, care să orchestreze colectarea distribuită de informații din aria definită de utilizator.
\\
\\
Comunicarea cu aplicația "Surf" se realizează prin intermediul unui API RESTful\footnote{\restDescription} construit pe platforma AWS\footnote{\awsDescription} API Gatway. Autentificarea utilizatorilor se va realiza printr-un serviciu OpenID Connect (e.g. Google, Facebook etc.). Autorizarea va avea ca principală componentă AWS IAM. Utilizatorilor le vor fi repartizate, în funcție de privilegiile asociate cu cheia de autentificare, o serie de roluri (i.e. drepturi de access asupra resurselor din cadrul serviciului "Surf"). Execuția codului propriu-zis, găzduit de funcții AWS Lambda, va interacționa cu serviciul pentru baze de date NoSQL AWS DynamoDB pentru a permite accesul la informații cheie pentru funcționalitatea aplicației (metadate crawling, date acces utilizatori etc.). Mediul de procesare distribuită va fi susținut de AWS Step Functions și configurat după preferințele utilizatorului. Datele extrase din procesul de web-crawling vor fi salvate în mediul persistent de stocare AWS S3.
